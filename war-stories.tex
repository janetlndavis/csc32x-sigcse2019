\section{War stories}

Of course, not everything in the course goes smoothly.  In this section,
we describe some illustrative problems we've encountered and suggest
some lessons each reveals. 

\subsection{Revealing an AWS key}

We ask Students to use Git to manage their projects.  While many
have prior experience with Git, the course
frequently expands their understanding and usage of Git; most move
from a single branch to using multiple branches, and from pushing
to sending pull requests.  Because many students had learned Git
informally and because the Internet is full of mediocre advice,
some students employ poor Git practices.  For example, there is a
group of students who frequently write ``\texttt{git add .; git
commit}.\footnote{That is, add any new or changed files in or below
the current directory and commit those files to the repository.}''
As one might expect, that leads to some inappropriate materials in
their public repositories.

We discovered one such example indirectly.  The instructor received
a call from Amazon.com asking, ``Do you realize that \$5,000 has
been charged to your AWS account in the past two days?''\footnote{During
development, the instructor had allowed the students to set up an
AWS account and to use his credit card.}  It appears that students
joining a project in the second semester had ignored their predecessor's
instructions and had put the AWS key in a public repository.  Since
there some applications regularly scrape public repositories for such
keys, it took little time for the now-public key to be found and
used.\footnote{Fortunately, an Amazon employee noted that Amazon
had also found the key in the repository, was sympathetic, and
waived the charge.}

That experience proved instructive not just for that semester's
students but also for subsequent students.  Students learned not
only that there are consequences when the inappropriately reveal
information, but also that such experiences are not uncommon.  The
more important lessons were about the technical and nontechnical
issues involved in keeping some information private in a project
that is primarily public.  Students needed to figure out technical
issues, such as how to update the \texttt{.gitignore} file to ensure
that the key would not be included, even by someone new to the
project who follows poor Git practices, how to configure the system to read
that key from an environment variable, and how to set environment
variables in a platform like Heroku.  But they also needed to decide
on practices.  For example, how do you share a private key among
group members without accidentally revealing it to others?  While
there are some common practices, and any organization they joined
would have standards, we saw value in letting students think through
the issue on their own and compare alternative strategies.

\subsection{Revealing personal information}

Lazy Git practice has also created other problems.  One group, while
developing an online directory for a local retirement community,
accidentally included photographs of all the residents in their
public GitHub repository.  In addition to the obvious privacy issues,
adding a large number of photographs also significantly increases
the download time for cloning the repository.

Fortunately, we caught this problem before any personal information
was revealed.\footnote{It helps that the photos were all in a single
zip file.}  Students had to learn how to ``scrub'' a GitHub repository:
beyond removing the photographs from the repository, it was essential
for them to remove the photographs from the repository's history.

This experience highlighted an issue that students regularly struggle
with: The overall structure of the project should be public so that
others can adapt or adopt it, but some portions need to be custom
for the client.  How do you appropriately separate the two?
Admittedly, this remains an issue that students struggle with; too
often, they consider the two aspects of the project together, rather
than separately.  We continue to draw on the example in helping them
reflect on this design issue.

\subsection{Uncooperative IT departments}

Unfortunately, the SMS service was not the only instance in which
students ended the project with a sense of disappointment.  We have
had at least two instances in which students finished a project for
a larger non-profit with its own IT department and then discovered
that the IT staff was less uncooperative than we expected.  In one
case, the software effectively disappeared after it was passed to
the client's IT department.  They thanked us, but did not appear
to have the time or enthusiasm to install it.  In the second case,
the IT staff agreed to provide a server for the students to install
software but took two-months of conversation to open port 80 on
that server.  After more than two months of back-and-forth
conversations, students were left with insufficient time to release
the software.

In both cases, we were able to help students focus on positive
outcomes.  While the projects did not get released, they had some
confidence that they would benefit others by releasing their software
with an open license And, in both cases, it was clear that their
regular client meetings had evolved partners' thinking about their
projects and processes.

We continue to explore ways to handle this broader issue of
transitioning the project to partner organizations.  In most cases,
it appears that using a common cloud platform, such as Heroku,
serves both the students and the partner organizations well.

\subsection{Changes in management}

Through the years, management of our partner organizations has
sometimes changed.  Often, the new manager has a very different
view of what the project should do.  In one case, the initial project
request was for a group text-messaging system with a Web interface.
We'd been surprised by the original project requirements since
free software with similar functionality exists.  However, the
client felt that they had particular enough requirements that a custom
solution was necessary.  When management changed, the usage model
also changed.  Instead of a Web interface, the client now wanted
what was essentially a text-message expander; they would text to
one number and have that text broadcast to a variety of other
numbers.  Students met the change with good nature; they arranged
for a phone number, identified and incorporated SMS receipt libraries,
explored security issues, such as how to limit who could use the
service, and designed a protocol to make it easy for the client to
select the recipient list.  Then management changed again.  And the
new manager decided that free software did a good enough job.

This situation was particularly challenging to help students through.
Since the organization's personnel changed, we could not emphasize
the broader benefits, ``Even though they're not using your software,
you've helped the organization grow in understanding.'' Because
similar free software existed, students did not think that there
would be a benefit to releasing their work as free-and-open-source
software.  In the end, we focused on what the students had learned
from the project and on ways they could share that knowledge with
future students who might need to incorporate SMS in future projects.

