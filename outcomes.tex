\section{Outcomes}

We have seen a variety of positive outcomes from the redesign,
many due to the way the various components interact.
As we've already mentioned, the design provides clearer
connections to their post-college careers and work experiences.
Students also feel more committed to these projects
because they understand the potential impact.  More broadly, they
understand that the work they can do as computer scientists can
have real and meaningful value to other people.

Perhaps the most important learning we see in the revised class is
that of soft skills.  Because they must regularly meet with a
non-technical partner, understand the needs of the partner, and
describe their own work (and, in some cases, the reasons for their
lack of progress), they greatly expand their communication skills.
As is often the case with small teams, they develop better interpersonal
skills.  Because we ask them to reflect on and compare their personal
strengths at the start of the semester,
%(originally with tools like Strength Finder, now a bit less formally), 
they think more broadly
about those skills and about the skills necessary for a successful
project.  
%Certainly, as different students have different priorities,
%some feel frustrated by teammates who they feel are spending
%too little (or too much) time on the project, but they also
%begin to learn how to navigate those complexities.  
Because we do
not require that projects be finished in the semester and because
we tell students to time-box their work, they not only feel
less stress, but are more willing to agree upon appropriate levels
of ambition. 

As we've noted, Ruby on Rails has proven both a strength and a weakness
of the course.  By the end of the semester, when they've started to develop
a better understanding of the Rails model, many realize that they could
could put together a small, database-backed, Web application
in about a week of full-time work, at least if everything goes as
expected.  
That realization gives them a sense of self-efficacy
and accomplishment.  At the same time, most also understand that if things
don't go right, and they often don't, there are additional challenges
to face.

We hope that these kinds of challenges help them develop broader
skills in learning how to learn.  Most are used to situations in
which the answers can be found in a textbook, from the instructor,
or, too frequently, on Stack Overflow.  Doing design and facing 
unexpected problems leads students to situations in
which they can't immediately find an answer.  What do you do when
Stack Overflow doesn't have an answer?  They must develop skills
at debugging (and, ideally, test-driven design that helps them
debug), in searching on both the web and on sites like Stack Overflow
to find relevant answers, in analyzing the applicability of advice
that doesn't quite match their problem, and in adapting other
solutions to their own situation.  Some clearly develop these kinds
of skills.  Others do not.

We also hope that these projects help students develop more of a
tolerance of ambiguity.  Most are accustomed to assignments with
clear requirements and find themselves frustrated that these 
assignments are open ended and, in many cases, they must develop
their own requirements or, more precisely, translate their clients'
requirements into their own requirements.

Unfortunately, we have not seen as many positive effects of
end-of-course evaluations as we had hoped.  While some students
describe the course as the center of their CS education, others
seem to find it pointless.  As one student puts it, ``In this course,
more than most, what you get out of this course depends directly
on what you put into the course.''  We are still exploring ways to
help students better understand and embrace this issue.

% Add a final paragraph?

