\section{Conclusion and recommendations}

By using a variety of traditional (e.g., real-world projects, agile
practices) and novel components (e.g., alumni mentors, multi-semester
projects, collaboration with the career development office), we
give students more realistic experiences in software design.  These
experiences develop not only technical skills but the less quantifiable
soft skills that are essential to their future success.

While we find the combination of course design components particularly
successful, others may find it useful to adapt just a few.  Recommendations
associated with some of the more novel aspects of the course follow.

%\subsection{Group formation}

% Can probably drop this section. 
% Yeah, I agree.

%Like most CS faculty, we have tried a wide variety of approaches
%to group formation.  We began by including Myers-Briggs styles or
%Strengths Finder strengths to build more balanced groups.  We found
%that the approach generally complicated the process and did not
%show sufficient benefit for the time spent.  We initially asked
%students list ``anti-preferences'' for partners, rather than
%preferences.  This approach sometimes led to some students being
%ostracized based on reputation rather than for valid reasons.  We
%have discontinued that practice.  However, we do make it clear to
%students that they can make special requests based on past history
%with another student; such requests are rare.
%
%We currently ask students to rank the available class projects in
%the second or third week of class.  We do a first pass in which we
%simply assign students to their first choice.  Working with the
%student class mentor or another student leader, 
%we then look at group size and composition.  Student leaders
%often have good insight into personalities.  We pay particular
%attention to issues of representation; we generally try to avoid
%groups with only one woman or only one man.  But we also consider
%issues of experience, trying to mix students taking the course early
%in their careers with those later in their careers.

\subsection{Identifying alumni mentors}

We have found LinkedIn to be a particularly useful tool for identifying
alumni mentors.  A search for people who attended the institution and
who had experience with Ruby on Rails produced a small, but sufficiently
large, list.  Discussions with other faculty in the department helped
us choose who to prioritize.  The response rate to invitations was high;
almost everyone who was invited accepted.

%There are certainly a host of other possible approaches.  One might search
%a departmental alumni database, if one is available.  One might broadcast
%a message to a mailing list of departmental alumni, although that approach
%might have the disadvantage of some alumni feeling disheartened that they
%volunteered and were not selected.  They might be invited to serve as
%readers of the projects or to provide backup advice.\footnote{For example,
%please monitor this Slack channel for general questions on Ruby, Rails,
%and object-oriented design.} %Have you actually done this?
%% This paragraph could be cut.

We have found it easiest to use the same cohort of alumni each semester.
The alumni have developed a deeper understanding of the course and what
students are and are not capable of.  Alumni mentors also provide some 
institutional memory for ongoing projects as students come and go. 
%Others may find benefit in a rotating group of alumni mentors.
% Last sentence could be cut.

\subsection{Partnering with career development}

We have found it productive to partner with our career development
office.  The training they give students before meeting with community
partners and the guidance that they give throughout the semester
are invaluable.  For those reasons, we highly recommend that you
partner with your corresponding office.

If you are unable to find an appropriate partner, it is certainly
possible to run your own ``preparing to meet with your community
partner'' session.  As we suggest in the section on our work with
career development, it can be useful to help students get a broader
sense of issues in the community, it is valuable to help students
unpack their own biases about the community and to explore the
likely biases the community has about them, and it is essential to
help them figure out the non-technical questions to discuss in the
first meeting with the client.  You will find it useful to brainstorm
with the students about to prepare for those kinds of questions.
For example, having them consider the different reasons they might
need to contact the client helps them prepare appropriate questions
for that meeting.

\subsection{Identifying community partners}

Getting the right partners is important.  You need projects that
are of the right definition, scope, and scale, useful but not too sensitive 
or mission-critical.
You also need a person who will understand
the issues associated with working with a student development team and
who is willing to accept a slow development process.  You learn many of
these things in preliminary meetings with potential partners; we recommend
having such meetings in the prior semester, if at all possible, and allowing 
time for meetings with organizations that do not lead to potential projects.

We are fortunate to have an outreach coordinator who helps with these kinds
of issues.  Among other things, they are careful to make sure that neither
side overcommits.  Our coordinator is particularly good at saying ``that
seems like too much'', particularly when we discuss the scale of the project
or the time commitments on both sides.  They have also been useful in
helping the partner understand possible projects and accept the limitations
of working with students.

If your institution has someone in this role, we encourage you to work with
them.  If you don't have an outreach coordinator, it may be helpful to
bring an extra person to meetings to provide some similar context.  Someone
who listens carefully and cautions people on over-committing is particularly
useful.

\subsection{Time-boxing}

We highly recommend timeboxing, both of individual tasks and of students' 
total hours of work on the project each week.
%In assessing students' time management, we relied primarily on
%weekly individual reports (annotated time logs) and group reports
%(tasks planned, tasks accomplished obstacles).  We also looked at the
%details of their code and processes, such as whether they were using
%appropriate pull requests.
%
That said, 
time-boxing is strange to most students.  They are used to assignments 
which demand a particular
product meeting detailed requirements.  You support them in embracing the
time-box model by providing clear process and quality expectations and
giving regular feedback, particularly early in the semester.

\subsection{Contextualizing the course}

As we've suggested earlier, not every student embraces the course.  It
is therefore essential to provide appropriate context for the course at
the beginning of the semester.  Explain to students what you expect them
to get out of the course.  For example, we focus on the benefits of learning
soft skills, of making a difference in the community, of encountering
more realistic situations, of developing the ability to put together a
real product, and of recovering from difficulties.
