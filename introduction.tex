
\section{Introduction}

Software design and software engineering form a core part of
the undergraduate computer science curriculum. However, perspectives
on the content and structure of software design courses vary
significantly.  Some focus on formal methods (e.g., \cite{liu-2009},
\cite{garcia-2014}).  Some consider design strategies such as design
patterns, while others compare plan-and-document with agile
methodologies \cite{gestwicki-2018}.  Some excite students through
the use of video games (e.g., \cite{wolz-2007}).  Many courses are
now adopting an emphasis on service to the community~\cite{hfoss-2018},
such as the Posse group's work on humanitarian free and open-source
software (HFOSS)~\cite{posse-2018}.
The role of projects also varies significantly.  Some curricula
place a small project in the software design or software engineering
course and then follow with a capstone project.  For others, the
project is the center of the software design course, with everything
else done in support of it. In either model, it can be difficult
to scope projects so that students can successfully complete them
in one term or even one academic year.

In this paper, we reflect on the curriculum model that we have
employed for the past four years.  The model introduces some novel
approaches, including (a) community non-profit organizations who
serve as project clients; (b) multi-semester projects that students
often join in the middle; (c) two separate half-courses, one that
introduces the principles and practices necessary for the design
of a medium-scale project and one that focuses the projects themselves;
(d) alumni project mentors who help guide project teams; (e) a
partnership with our career development office; (f) an emphasis on
Web-based software with a common platform; and (g) the use of a
MOOC/SPOC to ground learning.  At the same time, the course relies
on many traditional components, including a focus on agile methodologies
and object-oriented design.  Together, these many parts provide
students with a strong background in software design and, as
importantly, to build equally valuable soft skills.

In section 2 of this paper, we describe the institutional and
departmental context for the curriculum.  In section 3, we explore
the goals and design of the new software design component.  In
section 4, we report on selected ``war stories'' that have contributed
to student learning in the course and that others may find instructive.
In section 5, we describe some outcomes of the course.  Finally,
in section 6, we provide recommendations for others interested in
adapting some or all of these practices.

