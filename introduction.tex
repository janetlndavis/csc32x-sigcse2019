
\section{Introduction}

While software design and software engineering form a core part of
the undergraduate computer science curriculum, perspectives on 
content vary significantly.  Some courses
focus on formal methods (e.g., \cite{liu-2009}, \cite{garcia-2014}).
Some consider design strategies such as design patterns, while others
compare plan-and-document with agile methodologies
\cite{gestwicki-2018}.  Some excite students through the use of
video games (e.g., \cite{wolz-2007}).  Many courses are now adopting
an emphasis on service to the community~cite{hfoss-2018}, 
such as the Posse group's work on
humanitarian free and open-source software (HFOSS) \cite{posse-2018}.

Curricula also vary in the role of projects.  Some curricula place
a small project in the software design or software engineering
course and then follow with a capstone project.  For others,
the project is the center of the software design course, with
everything else done in support of it. In either model, it can be difficult to 
scope projects that students can successfully complete 
in one term or even one academic year.

In this paper, we explore the software design model that we have
meployed for the past four years.  That model, while including many
common characteristics of traditional software design courses, also
introduces some imporant innovations.  These innovations include
(a) community non-profit organizations serve as the primary client
base for the projects,
(b) multi-semester projects that students often join in the middle,
(c) separate half-courses, one that introduces
the principles and practices necessary for the design of a medium-scale
project, the other of which focuses primarily on the projects, 
(d) alumni project mentors, who help guide project teams, 
(e) a partnership with our career-development office, 
(f) an emphasis on Web-based software with a common platform, and 
(g) in many offerings, use of a MOOC/SPOC to ground learning.

These characteristics fit together in a way that not only provides
most students with a strong background in software design, but also
helps build a variety of soft skills that are likely to have at
least as much benefit for our students.

In section 2 of this paper, we describe the institutional and
departmental context for the curriculum.  In section 3, we explore
the goals and design of the new software design component.  In
section 4, we report on selected ``war stories'' that others may
find instructive.  In section 5, we describe some outcomes of the
course.  In section 6, we consider some of the changes of the course
over four years..  Finally, in section 7, we provide recommendations
for others intersted in adapting some or all of these practices.

